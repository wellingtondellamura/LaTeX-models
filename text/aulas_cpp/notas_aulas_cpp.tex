\documentclass[12pt,a4paper,oneside]{book}
\usepackage[utf8]{inputenc}
\usepackage[brazil]{babel}
\usepackage{amsmath}
\usepackage{amsfonts}
\usepackage{amssymb}
\usepackage{hyperref}
\usepackage{listings}
\usepackage{graphicx}
\usepackage[left=2.00cm, right=2.00cm, top=2.00cm, bottom=2.00cm]{geometry}
\author{Wellington Della Mura}
\title{Notas de Aula: Introdução à Programação}
\begin{document}
	\maketitle

\chapter{Conceitos Básicos}

\begin{flushright}
	\textit{``A essência do conhecimento consiste \\ em aplicá-lo, uma vez possuído.''} \\
	\textbf{Confúcio}
\end{flushright}

\section{Preparação do Ambiente}
Uma tarefa importante ao iniciar os estudos em programação é preparar o ambiente de desenvolvimento.
Para isso é preciso instalar os recursos necessários para a criação e edição do código-fonte e o compilador para gerar os programas na plataforma escolhida.
Este texto cobre o uso da linguagem C++\footnote{C++14 - O padrão atual da linguagem C++, oficialmente conhecido como \textit{ISO/IEC 14882:2014(E)} e disponível em \url{https://isocpp.org/std/the-standard}.} sobre plataformas baseadas em Debian GNU/Linux.

\subsection{Instalação do Compilador}
O compilador mais difundido nas plataformas baseadas em GNU/Linux é o GCC.
Na realidade o GCC\footnote{Mais informações sobre o GCC em \url{https://gcc.gnu.org/}.} é uma coleção de compiladores e bibliotecas para a construção de programas em linguagem C, C++, Objective C, Java, Fortran, entre outros.
É muito comum que os pacotes para utilizar o GCC já venham instalados por padrão na maioria das distribuições baseadas em Unix.
Para instalar o GCC a partir do código-fonte disponível no site oficial, o site \url{https://gcc.gnu.org/install/} provê todos os detalhes.
Contudo, para instalar os pacotes em distribuições baseadas em Debian basta seguir os comandos abaixo:

\begin{lstlisting}
# apt-get update
# apt-get install build-essential
\end{lstlisting}


\section{Compilando o primeiro programa}

\lstset{language=C++}
\begin{lstlisting}
#include <iostream>

int main(){
    std::cout << "Ola Mundo" << std::endl;
    return 0;
}
\end{lstlisting}


\end{document}