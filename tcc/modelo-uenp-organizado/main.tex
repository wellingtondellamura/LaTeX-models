%Este modelo possui as configurações mais comuns para um TCC.
%Na primeira linha são definidas as configurações principais do documento onde:
%- 12pt indica o tamanho da fonte
%- openright indica que todos os capítulos se iniciam nas folhas da direita
%- twoside indica que o texto será impresso frente e verso (o contrário é oneside)
%- a4paper determinar o tamanho do papel (pode ser utilizado letter se for necessário)
%- tcc indica que o trabalho se trata de um tcc e portanto todo os dados pré-textuais 
%possuem essa informação. Também pode ser utilizado o valor projeto para indicar o projeto de tcc.
\documentclass[12pt,openright, twoside, a4paper, tcc]{cct-uenp}
% PACOTES
\input{auxiliar/packages}

%informações
\input{auxiliar/info}

\makeindex % compila o indice

\begin{document}

% Retira espaço extra obsoleto entre as frases.
\frenchspacing 
% ----------------------------------------------------------
% ELEMENTOS PRÉ-TEXTUAIS
% ----------------------------------------------------------

% Capa (elemento obrigatório)
\imprimircapa 

% Folha de rosto (elemento obrigatório) % (o * indica que haverá a ficha bibliográfica)
\imprimirfolhaderosto* 

%ficha catalográfica (elemento obrigatório apenas se for para a biblioteca)
%\input{fichacatalografica.tex] 

% Folha de aprovação (elemento obrigatório)
\imprimirfolhadeaprovacao % ou \includepdf{folhadeaprovacao_final.pdf}

%dedicatória, agradecimentos e epígrafe(elemento obrigatório)
% ---
% Dedicatória (elemento opcional)
% ---
\begin{dedicatoria}
   \vspace*{\fill}
   \centering
   \noindent
   \textit{ Este trabalho é dedicado às crianças adultas que,\\
   quando pequenas, sonharam em se tornar cientistas.} \vspace*{\fill}
\end{dedicatoria}
% ---

% ---
% Agradecimentos (elemento opcional, mas fortemente recomendado)
% ---
\begin{agradecimentos}
Os agradecimentos principais são direcionados à Gerald Weber, Miguel Frasson,
Leslie H. Watter, Bruno Parente Lima, Flávio de Vasconcellos Corrêa, Otavio Real
Salvador, Renato Machnievscz\footnote{Os nomes dos integrantes do primeiro
projeto abn\TeX\ foram extraídos de
\url{http://codigolivre.org.br/projects/abntex/}} e todos aqueles que
contribuíram para que a produção de trabalhos acadêmicos conforme
as normas ABNT com \LaTeX\ fosse possível.

Agradecimentos especiais são direcionados ao Centro de Pesquisa em Arquitetura
da Informação\footnote{\url{http://www.cpai.unb.br/}} da Universidade de
Brasília (CPAI), ao grupo de usuários
\emph{latex-br}\footnote{\url{http://groups.google.com/group/latex-br}} e aos
novos voluntários do grupo
\emph{\abnTeX}\footnote{\url{http://groups.google.com/group/abntex2} e
\url{http://abntex2.googlecode.com/}}~que contribuíram e que ainda
contribuirão para a evolução do \abnTeX.

\end{agradecimentos}
% ---

% ---
% Epígrafe (elemento opcional)
% ---
\begin{epigrafe}
    \vspace*{\fill}
	\begin{flushright}
		\textit{``Não vos amoldeis às estruturas deste mundo, \\
		mas transformai-vos pela renovação da mente, \\
		a fim de distinguir qual é a vontade de Deus: \\
		o que é bom, o que Lhe é agradável, o que é perfeito.\\
		(Bíblia Sagrada, Romanos 12, 2)}
	\end{flushright}
\end{epigrafe}
% ---

%resumos (elemento obrigatório)
\input{pretextual/resumos} 

%lista de figuras, tabelas, símbolos, etc. (elemento opcional)
\input{pretextual/listas}  

%sumário (elemento obrigatório)
\pdfbookmark[0]{\contentsname}{toc} \tableofcontents* \cleardoublepage  

% ----------------------------------------------------------
% ELEMENTOS TEXTUAIS
% ----------------------------------------------------------
\pagestyle{uenp-header} % Configura cabeçalho para apresentar apenas números de página

% ----------------------------------------------------------
% Capitulos
% ----------------------------------------------------------
\input{textual/introducao}

\input{textual/fundamentacao}

\input{textual/metodologia}

\input{textual/desenvolvimento}

\input{textual/cronograma}

\input{textual/resultados}

\input{textual/conclusao}

% ----------------------------------------------------------
% ELEMENTOS PÓS-TEXTUAIS
\bibliographystyle{abntex2-alf} 
\bibliography{bibtex/bibliografia} % Referências bibliográficas

%\input{postextual/glossario}

%\input{postextual/apendices}

%\input{postextual/anexos}

%\input{postextual/publicacoes}

%\input{postextual/indice}

\end{document}
