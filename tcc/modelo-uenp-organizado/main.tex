%Este modelo possui as configurações mais comuns para um TCC.
%Na primeira linha são definidas as configurações principais do documento onde:
%- 12pt indica o tamanho da fonte
%- openright indica que todos os capítulos se iniciam nas folhas da direita
%- twoside indica que o texto será impresso frente e verso (o contrário é oneside)
%- a4paper determinar o tamanho do papel (pode ser utilizado letter se for necessário)
%- tcc indica que o trabalho se trata de um tcc e portanto todo os dados pré-textuais 
%possuem essa informação. Também pode ser utilizado o valor projeto para indicar o projeto de tcc.
\documentclass[12pt,openright, twoside, a4paper, tcc]{cct-uenp}
% PACOTES
\usepackage[T1]{fontenc}
\usepackage[utf8]{inputenc}
\usepackage{graphicx}
\DeclareGraphicsExtensions{.pdf,.png,.jpg,.ps,.fig,.eps}
\usepackage[usenames,dvipsnames]{xcolor}
\usepackage{epstopdf}
\usepackage{latexsym}
\usepackage{ae}
\usepackage{hyperref}
\usepackage{array}
\usepackage{tipa}
\usepackage{amssymb}
\usepackage{amsfonts}
\usepackage{booktabs}
\usepackage{dsfont}
\usepackage{textcomp}
\usepackage{cmll}
\usepackage{amsmath}
\usepackage{multirow}
\usepackage{multicol}
\usepackage[portuguese,lined,boxed,commentsnumbered,ruled]{algorithm2e}
\usepackage{stmaryrd}
\usepackage{listings}
\usepackage{framed}
\usepackage{lscape}
\usepackage{pdfpages}
\usepackage{tikz}
\usetikzlibrary{shapes,arrows,automata,positioning}



%Interessante para remover a cor dos links
%\hypersetup{
%	colorlinks,
%	linkcolor={black},
%	citecolor={black},
%	urlcolor={black}
%}

%informações
\titulo{Título do Trabalho}
\tituloingles{Title of the Work}
\palavraschave{Latex. Template ABNT. Editoração de texto.}
\palavraschaveingles{Latex. ABNT. Text editoration.}
\autor{Nome do(a) Aluno(a)}
\citacaoautor{SOBRENOME, N. A.}
\data{2017}

\diadefesa{24 de novembro}
\orientador{Prof(a). Dr(a). Nome do(a) Orientador(a)} % É membro nato e presidente da Banca Examinadora
% \coorientador{Prof(a). Dr(a). Nome do(a) Coorientador(a)} % Pode ou não ser membro da Banca; se for, deve ser incluído como membro a seguir
\membrobancadois{Prof. Dr. Segundo Membro da Banca}
\instmembrobancadois{Universidade/Instituição do Segundo Membro da Banca}
\membrobancatres{Prof. Dr. Terceiro Membro da Banca}
\instmembrobancatres{Universidade/Instituição do Terceiro Membro da Banca}
\membrobancaquatro{Prof. Ms. Quarto Membro da Banca}
\instmembrobancaquatro{Universidade/Instituição do Quarto Membro da Banca}


\makeindex % compila o indice

\begin{document}

% Retira espaço extra obsoleto entre as frases.
\frenchspacing 
% ----------------------------------------------------------
% ELEMENTOS PRÉ-TEXTUAIS
% ----------------------------------------------------------

% Capa (elemento obrigatório)
\imprimircapa 

% Folha de rosto (elemento obrigatório) % (o * indica que haverá a ficha bibliográfica)
\imprimirfolhaderosto* 

%ficha catalográfica (elemento obrigatório apenas se for para a biblioteca)
%\input{fichacatalografica.tex] 

% Folha de aprovação (elemento obrigatório)
\imprimirfolhadeaprovacao % ou \includepdf{folhadeaprovacao_final.pdf}

%dedicatória, agradecimentos e epígrafe(elemento obrigatório)
% ---
% Dedicatória (elemento opcional)
% ---
\begin{dedicatoria}
   \vspace*{\fill}
   \centering
   \noindent
   \textit{ Este trabalho é dedicado às crianças adultas que,\\
   quando pequenas, sonharam em se tornar cientistas.} \vspace*{\fill}
\end{dedicatoria}
% ---

% ---
% Agradecimentos (elemento opcional, mas fortemente recomendado)
% ---
\begin{agradecimentos}
Os agradecimentos principais são direcionados à Gerald Weber, Miguel Frasson,
Leslie H. Watter, Bruno Parente Lima, Flávio de Vasconcellos Corrêa, Otavio Real
Salvador, Renato Machnievscz\footnote{Os nomes dos integrantes do primeiro
projeto abn\TeX\ foram extraídos de
\url{http://codigolivre.org.br/projects/abntex/}} e todos aqueles que
contribuíram para que a produção de trabalhos acadêmicos conforme
as normas ABNT com \LaTeX\ fosse possível.

Agradecimentos especiais são direcionados ao Centro de Pesquisa em Arquitetura
da Informação\footnote{\url{http://www.cpai.unb.br/}} da Universidade de
Brasília (CPAI), ao grupo de usuários
\emph{latex-br}\footnote{\url{http://groups.google.com/group/latex-br}} e aos
novos voluntários do grupo
\emph{\abnTeX}\footnote{\url{http://groups.google.com/group/abntex2} e
\url{http://abntex2.googlecode.com/}}~que contribuíram e que ainda
contribuirão para a evolução do \abnTeX.

\end{agradecimentos}
% ---

% ---
% Epígrafe (elemento opcional)
% ---
\begin{epigrafe}
    \vspace*{\fill}
	\begin{flushright}
		\textit{``Não vos amoldeis às estruturas deste mundo, \\
		mas transformai-vos pela renovação da mente, \\
		a fim de distinguir qual é a vontade de Deus: \\
		o que é bom, o que Lhe é agradável, o que é perfeito.\\
		(Bíblia Sagrada, Romanos 12, 2)}
	\end{flushright}
\end{epigrafe}
% ---

%resumos (elemento obrigatório)
% ---
% RESUMOS
% ---

% ---
% Resumo em Português (elemento obrigatório)
% ---
\begin{resumo}
 Segundo a \citeonline[3.1-3.2]{NBR6028:2003}, o resumo deve ressaltar o
 objetivo, o método, os resultados e as conclusões do documento. A ordem e a extensão
 destes itens dependem do tipo de resumo (informativo ou indicativo) e do
 tratamento que cada item recebe no documento original. O resumo deve ser
 precedido da referência do documento, com exceção do resumo inserido no
 próprio documento. (\ldots) As palavras-chave devem figurar logo abaixo do
 resumo, antecedidas da expressão Palavras-chave:, separadas entre si por
 ponto e finalizadas também por ponto.
\end{resumo}
% ---

% ---
% Resumo em Inglês (elemento obrigatório)
% ---
% O ambiente Abstract (com A maiúsculo) é definido no estilo dc-uel
\begin{Abstract}
 This is the english abstract. The Abstract in English should be faithful to the
 Resumo in Portuguese, but not a literal translation.
\end{Abstract}
% ---
 

%lista de figuras, tabelas, símbolos, etc. (elemento opcional)
% ---
% Lista de ilustrações (elemento opcional, mas fortemente recomendado)
% ---
\pdfbookmark[0]{\listfigurename}{lof}
\listoffigures*
\cleardoublepage
% ---

% ---
% Lista de tabelas (elemento opcional, mas fortemente recomendado)
% ---
\pdfbookmark[0]{\listtablename}{lot}
\listoftables*
\cleardoublepage
% ---

% ---
% Lista de abreviaturas e siglas (elemento opcional)
% ---
\begin{siglas}
  \item[ABNT] Associação Brasileira de Normas Técnicas
  \item[BNDES] Banco Nacional de Desenvolvimento Econômico e Social
  \item[IBGE] Instituto Nacional de Geografia e Estatística
  \item[IBICT] Instituto Brasileiro de Informação em Ciência e Tecnologia
  \item[NBR] Norma Brasileira
\end{siglas}
% ---

% ---
% Lista de símbolos (elemento opcional)
% ---
% \begin{simbolos}
%   \item[$ \Gamma $] Letra grega Gama
%   \item[$ \Lambda $] Lambda
%   \item[$ \zeta $] Letra grega minúscula zeta
%   \item[$ \in $] Pertence
% \end{simbolos}
% ---
  

%sumário (elemento obrigatório)
\pdfbookmark[0]{\contentsname}{toc} \tableofcontents* \cleardoublepage  

% ----------------------------------------------------------
% ELEMENTOS TEXTUAIS
% ----------------------------------------------------------
\pagestyle{uenp-header} % Configura cabeçalho para apresentar apenas números de página

% ----------------------------------------------------------
% Capitulos
% ----------------------------------------------------------
\chapter{Introdução}

Este documento e seu código-fonte são uma pequena adaptação do exemplo de uso
fornecido pela equipe desenvolvedora da classe \textsf{abntex2}, atendendo a
particularidades inseridas na classe \textsf{ABNT-DC-UEL.cls} que define
o modelo para Trabalhos de Conclusão de Curso e Dissertações de Mestrado dos
cursos do Departamento de Computação da Universidade Estadual de Londrina.
Observe-se que o documento está preparado para impressão em frente e verso
(opções twoside e openright) e que para gerar o índice remissivo deve-se 
utilizar o comando \textsf{makeindex}.
Sugestões de melhorias ou problemas encontrados, favor notificar (Prof.
Daniel S. Kaster -- \href{mailto:dskaster@uel.br}{dskaster@uel.br}).

Este documento e seu código-fonte são exemplos de referência de uso da classe
\textsf{abntex2} e do pacote \textsf{abntex2cite}. O documento 
exemplifica a elaboração de trabalho acadêmico (tese, dissertação e outros do
gênero) produzido conforme a ABNT NBR 14724:2011 \emph{Informação e documentação
- Trabalhos acadêmicos - Apresentação}.

A expressão ``Modelo Canônico'' é utilizada para indicar que \abnTeX\ não é
modelo específico de nenhuma universidade ou instituição, mas que implementa tão
somente os requisitos das normas da ABNT. Uma lista completa das normas
observadas pelo \abnTeX\ é apresentada em \citeonline{abntex2classe}.

Sinta-se convidado a participar do projeto \abnTeX! Acesse o site do projeto em
\url{http://abntex2.googlecode.com/}. Também fique livre para conhecer,
estudar, alterar e redistribuir o trabalho do \abnTeX, desde que os arquivos
modificados tenham seus nomes alterados e que os créditos sejam dados aos
autores originais, nos termos da ``The \LaTeX\ Project Public
License''\footnote{\url{http://www.latex-project.org/lppl.txt}}.

Encorajamos que sejam realizadas customizações específicas deste exemplo para
universidades e outras instituições --- como capas, folha de aprovação, etc.
Porém, recomendamos que ao invés de se alterar diretamente os arquivos do
\abnTeX, distribua-se arquivos com as respectivas customizações.
Isso permite que futuras versões do \abnTeX~não se tornem automaticamente
incompatíveis com as customizações promovidas. Consulte
\citeonline{abntex2-wiki-como-customizar} par mais informações.

Este documento deve ser utilizado como complemento dos manuais do \abnTeX\ 
\cite{abntex2classe,abntex2cite,abntex2cite-alf} e da classe \textsf{memoir}
\cite{memoir}. 

Esperamos, sinceramente, que o \abnTeX\ aprimore a qualidade do trabalho que
você produzirá, de modo que o principal esforço seja concentrado no principal:
na contribuição científica.

Equipe \abnTeX 

Lauro César Araujo


\input{textual/fundamentacao}

\input{textual/metodologia}

\input{textual/desenvolvimento}

\input{textual/cronograma}

\input{textual/resultados}

\input{textual/conclusao}

% ----------------------------------------------------------
% ELEMENTOS PÓS-TEXTUAIS

\bibliography{bibtex/bibliografia} % Referências bibliográficas

%% ----------------------------------------------------------
% Glossário
% ----------------------------------------------------------
%
% Consulte o manual da classe abntex2 para orientações sobre o glossário.
%
%\glossary


%% ----------------------------------------------------------
% Apêndices
% ----------------------------------------------------------
% ---
% Apêndices (elemento opcional)
%
% São textos ou documentos elaborados pelo autor do trabalho a fim complementar
% a sua argumentação.
% ---
\begin{apendicesenv}

% Imprime uma página indicando o início dos apêndices
\partapendices

% ----------------------------------------------------------
\chapter{Teste}
% ----------------------------------------------------------



\end{apendicesenv}
% ---


%% ----------------------------------------------------------
% Anexos (elemento opcional)
%
% São textos ou documentos, não elaborado pelo autor do trabalho que podem servir como
% ilustração, comprovação ou que contribua de forma relevante com o conteúdo já apresentado.
% ----------------------------------------------------------

% ---
% Inicia os anexos
% ---
\begin{anexosenv}

% Imprime uma página indicando o início dos anexos
\partanexos

% ---
\chapter{Teste Dois}
% ---


\end{anexosenv}


%% ----------------------------------------------------------
% Trabalhos publicados pelo autor
%
% ----------------------------------------------------------
\chapter*{Trabalhos Publicados pelo Autor}
\addcontentsline{toc}{chapter}{Trabalhos Publicados pelo Autor}

\noindent
Trabalhos publicados pelo autor durante o programa.

\vspace{12pt}

\begin{enumerate}

\item Jose da silva, autor2 da silva, orientador da silva, \textbf{Título do artigo}, local onde foi
publicado, mês/ano, editora, número de página, isbn, (Qualis CC 2012, xx)

\item Jose da silva, autor2 da silva, orientador da silva, \textbf{Título do artigo}, local onde foi
publicado, mês/ano, editora, número de página, isbn, (Qualis CC 2012, xx)

\item Jose da silva, autor2 da silva, orientador da silva, \textbf{Título do artigo}, local onde foi
publicado, mês/ano, editora, número de página, isbn, (Qualis CC 2012, xx)

\item Jose da silva, autor2 da silva, orientador da silva, \textbf{Título do artigo}, local onde foi
publicado, mês/ano, editora, número de página, isbn, (Qualis CC 2012, xx)

\end{enumerate}


%%---------------------------------------------------------------------
% INDICE REMISSIVO (elemento opcional)
%---------------------------------------------------------------------
% Requer incluir instruções \index{...} no decorrer do texto, para marcar os termos a serem indexados

\printindex

\end{document}
